\documentclass[10pt, oneside,english]{article}   	
\usepackage{geometry}
\geometry{a4paper}                   		
\usepackage[utf8]{inputenc}               		

\usepackage{caption}
\usepackage{subfigure}
\usepackage{graphicx}							
\usepackage{amssymb}
\usepackage{authblk}
\usepackage[backend=bibtex,maxbibnames=99]{biblatex}
\usepackage{bibentry}
\usepackage{enumitem}

\nobibliography*

\bibliography{../ListPublications/mypubs.bib}


\preto\fullcite{\AtNextCite{\defcounter{maxnames}{99}}}


\begin{document}

\begin{center}
{\huge{\bf List of Key Publications of \\ \vspace{0.1cm}
 Laura Leal-Taix{\'e}}}
\end{center}


\vspace{0.2cm}


\begin{center}
{\large{\it In chronological order}}
\end{center}

\vspace{1cm}


\begin{enumerate}


\item \fullcite{lealiccv2011}
\vspace{0.2cm}

One of the two main contributions of my PhD thesis, this paper is my most cited contribution and the one that has had a bigger impact in the community. 
In this work, I explored the use of Linear Programming for multiple object tracking. This methodology has since been established as the de-facto framework for this task. This was one of the first works to explore physical motion models for pedestrian tracking, a trend that is still an active area of research as of today. 
It laid the groundwork for my thesis and most of my postdoc work and got me extremely interested in {\bf how people interact with each other and the environment}, which is the driving force of this proposal. 

\vspace{0.6cm}


\item \fullcite{lealcvpr2012}
\vspace{0.2cm}


This work was my first contribution as first author in a major conference in computer vision (with around 25\% acceptance rate). In our field, researchers prioritize publishing in conferences like ICCV, CVPR and ECCV before Journals, since the field is rapidly evolving, and conferences provide a shorter review period and faster dissemination of the work. 
It was technically one of the most challenging papers I have ever written, as it required deep understanding of 3D geometry and optimization techniques.  In this work I explored the field of {\bf optimization} in detail, which is why I feel confident I can provide a high level of expertise in this proposal. 
The paper dealt with multiple object tracking in a multi-camera setting, formulating the problems of reconstruction and tracking in a single optimization function. It was the second main contribution of my PhD thesis.

\vspace{0.6cm}


\item \fullcite{lealcvpr2014}
\vspace{0.2cm}

This work was the first contribution as a postdoctoral researcher at ETH Zurich, closely following my PhD work. Following the work done at ICCV 2011, I realized that motion models and appearance could be better learned statistically rather than modeled using a few handcrafted terms. This is when I started delving deeper into the field of {\bf machine learning}. The goal was to learn the motion models proposed in the ICCV 2011 paper directly from image features.  This allowed the model to represent a wider range of motion interactions, improve tracking results and furthermore allow tracking to work in image space.  This is especially beneficial when the 3D structure of the scene cannot be inferred accurately, with potential impact in areas like autonomous driving, where pedestrian tracking is of paramount importance.
This work was followed up by a recent contribution we made at CVPR 2016, where we used deep learning to learn motion and appearance models that improved tracking even further.

\vspace{0.6cm}



\item \fullcite{milancvpr2015}
\vspace{0.2cm}

Second main work of my postdoc time at ETH Zurich, in this work I collaborated with a postdoctoral researcher from Adelaide in order to create a method that would perform multiple object tracking as well as instance segmentation. In particular, my contribution was to work on the appearance model to obtain accurate segmentations. This work was important for me to {\bf learn how to productively collaborate with researchers outside my group}, where communication is hard due to the distance and the time difference. 
The collaboration was so productive that the two groups joined forces to created the MOTChallenge tracking benchmark, an effort to standardize multiple object tracking evaluation and further push tracking research by providing highly accurate annotations of difficult crowded environments. The benchmark has since been established as the main tracking challenge dataset, with almost 70 publicly available results from state-of-the-art tracking methods.
The MOTChallenge work is currently being written into a Journal article.


\vspace{0.6cm}

\item \fullcite{fenziiccv2015}
\vspace{0.2cm}

During my PhD studies I collaborated with several colleagues in works on human and hand pose estimation. From all collaborations, I deem the work with M. Fenzi on vehicle pose estimation the most fruitful one. This work is the final contribution to a high ranked conference, where we explored the relationship between image features and viewpoint through regression. 
As a senior PhD student, I acted as a postdoc, advising M. Fenzi throughout his PhD and collaborating on the theoretical and practical level, producing a total of 5 papers together. I learned how to advise students, how to motivate them, and put together their ideas to create a novel and solid approach for a specific goal. This experience was {\bf key for my education as a scholar and group leader}. 





\end{enumerate}


%\nocite{*}
%\printbibliography

\end{document}   
